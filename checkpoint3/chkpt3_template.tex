\documentclass[12pt]{article}
\usepackage{fullpage} 
\usepackage{microtype}      % microtypography
\usepackage{array}
\usepackage{amsmath,amssymb,amsfonts}
\usepackage{amsthm}

%% Header
\usepackage{fancyhdr}
\fancyhf{}
\fancyhead[C]{CS 136 - 2022s - Checkpoint3 Submission}
\fancyfoot[C]{\thepage} % page number
\renewcommand\headrulewidth{0pt}
\pagestyle{fancy}

\usepackage[headsep=0.5cm,headheight=2cm]{geometry}

%% Hyperlinks always blue, no weird boxes
\usepackage[hyphens]{url}
\usepackage[colorlinks=true,allcolors=black,pdfborder={0 0 0}]{hyperref}

%%% Doc layout
\usepackage{parskip}
\usepackage{times}

%%% Write out problem statements in blue, solutions in black
\usepackage{color}
\newcommand{\officialdirections}[1]{{\color{blue} #1}}

%%% Avoid automatic section numbers (we'll provide our own)
\setcounter{secnumdepth}{0}

\begin{document}
~~\\ %% add vert space

{\Large{\bf Student Names: TODO}}


{\Large{\bf Collaboration Statement:}}

Turning in this assignment indicates you have abided by the course Collaboration Policy:

\url{www.cs.tufts.edu/comp/136/2022s/index.html#collaboration-policy}

Total hours spent: TODO

We consulted the following resources:
\begin{itemize}
\item TODO
\item TODO
\item $\ldots$	
\end{itemize}

\newpage

These are the official instructions for checkpoint 3.  You can find instructions on how to submit at \url{www.cs.tufts.edu/comp/136/2022s/checkpoint3.html}

\textbf{Please consult the full project description at \url{https://www.cs.tufts.edu/comp/136/2022s/project.html} in addition to this document when working on this checkpoint.  It gives details on what we expect.}

\section{Recap of your dataset, your model, and the issues you hope to address with your upgrade}

The goal of this section is to provide enough context for us to understand your proposed upgrade.  You should include a brief recap of your dataset; a brief recap of which model you are using; and a brief recap of the issues you have run into that you hope your upgrade will address.

This section should be at most 1/2 page.

\textbf{Subsection grading rubric:}
\begin{itemize}
	\item Recap dataset (3 points)
	\item Recap model (3 points)
	\item Recap problems for upgrade to address (3 points)
\end{itemize}

\section{Detailed description of your upgrade}

In this section, you should describe your upgrade in detail.  This should include a verbal description, any equations that you are implementing, and an outline of the procedure you will follow for implementing your upgrade in code.  The more detail you provide, the more feedback we can give you.

This section should be no more than 1 page total. 

\textbf{Subsection grading rubric:}
\begin{itemize}
	\item Verbal description of upgrade (5 points)
	\item Equations for upgrade (5 points)
	\item Description of procedure for implementing upgrade (5 points)
\end{itemize}

\section{Questions/Issues you want advice on}

In this section, you may (but aren't require to) fill out this bulleted list with questions you have that we can try to help with.  Please provide enough relevant detail.  For example, if gradient descent isn't converging, tell us what your convergence criteria is and what the behavior is (at the minimum).

\begin{itemize}
	\item Question 1: ...
\end{itemize}

\textbf{Subsection grading rubric: N/A}

\end{document}


